\chapter*{Conclusion}
\addcontentsline{toc}{chapter}{Conclusion}
\chaptermark{Conclusion}
\markboth{Conclusion}{Conclusion}



    In this thesis, we have seen four major constructions: Gentry's construction in Chapter 3, the DGHV construction in Chapter 4, and the BGV and GSW schemes of Chapter 5. Differing widely in hardness assumptions, complexity, and algebraic techniques, they, along with all other known constructions of FHE, share a commonality: they are forced to deal with noise present in ciphertexts. This is unfortunate, for it leads us to approaches that we would rather do away with, such as bootstrapping, large public keys, and leveled homomorphism.

    Indeed, noise reduction is the primary reason why FHE has a reputation for being inefficient. One of the most efficient implementations built so far of FHE is Halevi and Shoup's HElib, discussed in Section \ref{sec:bgvoptimizations}. In their system, a bootstrapping operation is required every $9$ gates, with an amortized bootstrapping cost of $310$ milliseconds per plaintext slot. This system is not nearly efficient enough for large-scale applications. For a large application, $500$ is a conservative lower bound on circuit depth; this results in at least $17$ seconds of computation time spent on this computation alone. This is far too slow to be deployed on a cloud service.

    There are still many open areas of research regarding the efficiency of HElib and whether it could be adapted for real use in low-depth use cases. Most notably, HElib has not yet been parallelized; the above figure of $310$ milliseconds was obtained by performing bootstrapping on one 2.93GHz core. In their work, Halevi and Shoup note that parallelizing and running tests on a more modern machine could yield optimistic results \cite{Halevi2015}.

    In Section \ref{sec: directapplication}, we discuss the work of Lauter et al., who consider the application areas of low-depth homomorphic computations (such as performing averages, standard deviations, etc) \cite{bvpractical}. At the time of writing, the most efficient homomorphic encryption scheme at hand was the BV scheme; it would be a good forward direction to review their work in light of the BGV and GSW schemes of Chapter \ref{chap: lwe}, and to see how practical their computations would be using the HElib encryption library.

    Another further direction would be to construct a robust implementation of the GSW encryption system. In \cite{Ducas2015}, Ducas et al.~display a homomorphic bootstrapping procedure for GSW that runs \emph{on a single plaintext slot} in just 0.69 seconds; i.e., bootstrapping in the GSW encryption system non-amortized performs similarly to HElib's bootstrapping procedure amortized over many plaintext slots. Because of this, it is reasonable to suggest that the GSW encryption system has more potential for speed than the BGV system. However, Ducas et al.~are working over single-bit plaintext spaces, while Halevi and Shoup are working over larger plaintext spaces. It is still open how efficient batching and Ring-LWE techniques would be for the GSW system.

    Now, we end the thesis with a provocation. Can we construct a \emph{noiseless} FHE scheme? That is, can we find a scheme which is \emph{intrinsically} fully homomorphic, and requires no noise analysis, bootstrapping, or predetermined level?

    All FHE schemes covered in this thesis, in some manner, rely on \emph{analytic} hardness assumptions. For example, we may ask for answers to relatively simple problems, modulo some error; the learning with errors problem and the approximate GCD problem are of this sort. Or, we might give a basis of some lattice, and ask for its \emph{smallest} lattice point, or its \emph{closest} lattice point to some specified vector. Certainly, these questions are also analytic.

    In 2015, Koji Nuida presented a new perspective on homomorphic encryption. He, in \cite{nuida15}, instead considers an \emph{algebraic} hardness assumption. Given a group $G$ isomorphic to $H \times N$ (given by a set of generators and relations, called a \emph{presentation}) the \emph{subgroup membership problem} asks whether a given group element $g \in G$ is sampled uniformly from $G$, or from $\{e\} \times N$, where $e$ is the identity element of $H$. (Below, we will abbreviate $\{e\} \times N$ as simply $N$, understanding that $N$ is a subgroup of $G$.)

    Given a group with such a property, one constructs an encryption scheme as follows: to encrypt a $0$, one samples two group elements $c_1 \leftarrow G$, $c_2 \leftarrow N$; to encrypt a $1$, one samples the two group elements $c_1 \leftarrow G$, $c_2 \leftarrow G$. Then, given the inclusion map $\pi: N \to G$, one decrypts by simply asking whether $c_2$ is in $N$. Homomorphic properties of this scheme follow, subject to a technical (but provable) condition on the group $G$, named \emph{commutator-separability}. The public key of this scheme is a method to sample from $G$ and $N$, and the secret key is the map $\pi$.

    Thus, we have three conditions placed on $G$: some description of $G$ and $N$ must be made public so that one may sample from them, the operation $\pi$ must not be efficiently computable from this description of $G$ and $N$, \emph{and} the subgroup membership problem must be hard for $G$ and $N$. (Necessarily, $G$ must not be commutative). It is \emph{extremely} difficult to find a group that meets these three conditions.

    Nuida's proposed solution is this: one ``garbles'' the group $G$ by performing a long series of \emph{Tietze transformations} on the presentation, which maps the original presentation $P$ of $G$ to a much more convoluted, hard-to-compute-with one, $P'$. (Thus, this new presentation $P'$ is analogous to the Hermite normal form of a lattice basis for lattice-based schemes.) When this transformation is performed, one records the transformation step-by-step, effectively creating a trapdoor which allows us to convert $P'$ back to $P$. This trapdoor is our secret key; once we make our way back to $P$, the mapping $\pi$ is trivially computable. The public key is the new presentation $P'$. Without the trapdoor, one hopes that it is infeasible to recover $\pi$ from $P'$. Furthermore, one hopes that the group $G$ and the subgroup $N$ are both sampleable from $P'$.

    The above truly are only hopes, since Nuida's work was only to present the above ideas. A detailed analysis of Nuida's proposal, and indeed noiseless FHE in general, are left as open questions for the time being.
