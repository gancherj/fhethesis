\chapter*{Abstract}
	\emph{Fully homomorphic encryption} (FHE) enables one to carry out computations using ciphertexts, rather than plaintexts. One may wish to compute the output of a boolean circuit $C$ given input bits $\{b_i\}$, in order to produce an output bit $b'$. With FHE, instead one can \emph{encrypt} each $b_i$ to create a ciphertext $c_i$, and run a corresponding computation ${\sf Eval}(C, \{c_i\})$ on the ciphertexts, in order to produce an output ciphertext $c'$. Then, one can decrypt $c'$ to retrieve $b'$.

    Crucially, a third party may run ${\sf Eval}$ using only the public key. By the security of the encryption scheme, this third party can run ${\sf Eval}$ to produce $c'$ out of the $c_i$, but not learn anything about $c'$ or any of the $c_i$ in the process. Only with the secret key can anybody learn about the contents of the ciphertexts.
    Using this technology, one may offload an arbitrary computation to somebody else without disclosing the contents of the data. Because of this, FHE is a highly applicable primitive. How to achieve FHE was a major open question until Craig Gentry's breakthrough 2009 thesis, which has since inspired years of cryptographic research.

    In this thesis, I survey this research. After outlining the relevant security definitions and hardness assumptions for FHE, I will present four major constructions of FHE schemes: Gentry's initial construction \cite{gentry2009fully}, the DGHV scheme over the integers \cite{dghv}, and two FHE schemes based off of the learning with errors problem \cite{bgv2011} \cite{gsw}. Afterwards, I will outline some applications of FHE.
